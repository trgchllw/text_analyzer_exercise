\documentclass[a4paper,12pt]{article}
\usepackage[utf8]{inputenc}
\usepackage[russian]{babel}
\usepackage{listings}
\usepackage{geometry}
\usepackage{graphicx}
\geometry{left=2.5cm,right=2.5cm,top=2cm,bottom=2cm}

\title{\textbf{Инструкция по установке Text Analyzer}}
\author{Студент группы ИТПМ-125 \\ Березин Евгений Андреевич}
\date{\today}

\begin{document}

\maketitle

\section{Предварительные требования}

Перед установкой убедитесь, что в системе установлены:

\begin{itemize}
    \item Компилятор GCC
    \item Утилита Make
    \item Система сборки LaTeX (опционально, для документации)
\end{itemize}

\subsection{Проверка наличия GCC}
\begin{lstlisting}[language=bash,frame=single]
gcc --version
\end{lstlisting}

\subsection{Проверка наличия Make}
\begin{lstlisting}[language=bash,frame=single]
make --version
\end{lstlisting}

\section{Сборка из исходников}

\subsection{Шаг 1: Скачивание исходного кода}
Скопируйте исходный код в директорию \texttt{text\_analyzer}.

\subsection{Шаг 2: Компиляция}
Перейдите в директорию проекта и выполните:
\begin{lstlisting}[language=bash,frame=single]
make compile
\end{lstlisting}
Результат: создание исполняемого файла \texttt{text\_analyzer}.

\subsection{Шаг 3: Генерация документации}
Если установлен LaTeX:
\begin{lstlisting}[language=bash,frame=single]
make docs
\end{lstlisting}
Результат: PDF файлы в директории \texttt{docs/}.

\subsection{Шаг 4: Конфигурация}
Создайте конфигурационный файл:
\begin{lstlisting}[language=bash,frame=single]
make configure
\end{lstlisting}
Скрипт запросит:
\begin{itemize}
    \item ФИО пользователя
    \item Email (проверяется формат)
    \item Телефон (проверяется формат, 10-15 цифр)
\end{itemize}
Результат: файл \texttt{install.conf} с сертификатом.

\subsection{Шаг 5: Установка}
Для установки в систему выполните:
\begin{lstlisting}[language=bash,frame=single]
sudo make install
\end{lstlisting}
В результате:
\begin{itemize}
    \item Программа копируется в \texttt{/usr/bin/text\_analyzer}
    \item Создается символическая ссылка \texttt{/usr/bin/analyzer}
    \item Сертификат копируется в \texttt{/etc/text\_analyzer/install.conf}
    \item Создается директория для логов \texttt{/var/log/text\_analyzer}
\end{itemize}

\section{Проверка установки}

После установки проверьте:
\begin{lstlisting}[language=bash,frame=single]
# Поиск программы
which text_analyzer

# Проверка прав
ls -l /usr/bin/text_analyzer

# Проверка сертификата
ls -l /etc/text_analyzer/install.conf
\end{lstlisting}

\section{Использование после установки}

Запуск программы без указания пути:
\begin{lstlisting}[language=bash,frame=single]
text_analyzer -i
\end{lstlisting}

\section{Обновление программы}

Для обновления на новую версию:
\begin{lstlisting}[language=bash,frame=single]
# Удаление старой версии
sudo make uninstall

# Установка новой
make compile
sudo make install
\end{lstlisting}

\section{Удаление программы}

Полное удаление программы:
\begin{lstlisting}[language=bash,frame=single]
sudo make uninstall
\end{lstlisting}

При необходимости удалить также логи:
\begin{lstlisting}[language=bash,frame=single]
sudo rm -rf /var/log/text_analyzer
\end{lstlisting}

\section{Решение проблем}

\subsection{Ошибка прав доступа}
Если при установке возникает ошибка Permission denied, используйте \texttt{sudo}.

\subsection{Ошибка компиляции}
Убедитесь, что установлен GCC и все заголовочные файлы стандартной библиотеки.

\subsection{Ошибка генерации PDF}
Установите LaTeX (для macOS):
\begin{lstlisting}[language=bash,frame=single]
brew install --cask mactex-no-gui
\end{lstlisting}

\section{Состав пакета}

После распаковки должна быть следующая структура:
\begin{lstlisting}[frame=single]
text_analyzer/
├── Makefile
├── LICENSE.txt
├── LICENSE.tex
├── install.conf (после configure)
├── src/
│   ├── main.c
│   ├── core/
│   │   ├── analyzer.h
│   │   └── analyzer.c
│   ├── ui/
│   │   ├── interface.h
│   │   └── interface.c
│   └── utils/
│       ├── logger.h
│       └── logger.c
├── docs/
│   ├── AUTHOR
│   ├── README
│   ├── INSTALL
│   └── *.tex
└── scripts/
    └── configure
\end{lstlisting}

\end{document}