\documentclass[a4paper,12pt]{article}
\usepackage[utf8]{inputenc}
\usepackage[russian]{babel}
\usepackage{listings}
\usepackage{graphicx}
\usepackage{geometry}
\geometry{left=2.5cm,right=2.5cm,top=2cm,bottom=2cm}

\title{\textbf{Руководство пользователя Text Analyzer}}
\author{Студент группы ИТПМ-125 \\ Березин Евгений Андреевич}
\date{\today}

\begin{document}

\maketitle

\section{Общие сведения}
Text Analyzer — утилита для анализа текстовых файлов, разработанная в рамках курсовой работы по дисциплине «Программирование».

\subsection{Возможности программы}
\begin{itemize}
    \item Подсчет количества строк, слов и символов в текстовом файле
    \item Поиск заданного слова в файле с подсчетом вхождений
    \item Ведение логов всех операций с временными метками
\end{itemize}

\section{Установка программы}

\subsection{Требования}
\begin{itemize}
    \item Операционная система: Linux, macOS
    \item Компилятор: GCC с поддержкой стандарта C99
    \item Дисковое пространство: минимум 10 МБ
\end{itemize}

\subsection{Процедура установки}
\begin{lstlisting}[language=bash,frame=single]
# 1. Компиляция
make compile

# 2. Конфигурация (создание сертификата)
make configure

# 3. Установка в систему
sudo make install
\end{lstlisting}

После установки программа доступна по команде \texttt{text\_analyzer}.

\section{Использование программы}

\subsection{Запуск в интерактивном режиме}
Запустите программу без параметров или с аргументом \texttt{-m}:
\begin{lstlisting}[language=bash,frame=single]
./text_analyzer -m
\end{lstlisting}

Появится меню:
\begin{verbatim}
=== Text Analyzer ===
1. Подсчет статистики файла
2. Поиск слова в файле
3. Выход
Выберите действие (1-3):
\end{verbatim}

\subsection{Запуск в пакетном режиме}
Для работы с конкретным файлом используйте аргументы:

\subsubsection{Подсчет статистики}
\begin{lstlisting}[language=bash,frame=single]
./text_analyzer -f input.txt -o stats.txt
\end{lstlisting}

Результат будет сохранен в файл \texttt{stats.txt}.

\subsubsection{Поиск слова}
\begin{lstlisting}[language=bash,frame=single]
./text_analyzer -f input.txt -w слово -o result.txt
\end{lstlisting}

\section{Аргументы командной строки}

Можно указать следующие параметры:

\begin{center}
\begin{tabular}{|l|p{8cm}|}
\hline
\textbf{Аргумент} & \textbf{Описание} \\
\hline
\texttt{-C <файл>} & Использовать конфигурационный файл \\
\hline
\texttt{-d} & Режим отладки (детальное логирование) \\
\hline
\texttt{-f <файл>} & Входной файл для обработки \\
\hline
\texttt{-h} & Показать справку \\
\hline
\texttt{-i} & Показать информацию об авторе \\
\hline
\texttt{-o <файл>} & Сохранить результат в файл \\
\hline
\texttt{-S} & Тихий режим (без логов) \\
\hline
\texttt{-t} & Не удалять временные файлы \\
\hline
\end{tabular}
\end{center}

\section{Пример работы}

\subsection{Интерактивное меню}
\includegraphics[width=\textwidth]{screenshot_menu.jpg}


\section{Логирование}

Программа ведет журнал операций в директории \texttt{/var/log/text\_analyzer/}.

\subsection{Структура лога}
\begin{lstlisting}[frame=single]
[2025-12-19 16:30:22] user@example.com: count_stats = Stats: L=4 W=8 C=35 | Duration=0s | Code=0
\end{lstlisting}

\subsection{Анализ логов}
Используйте \texttt{awk} для анализа:
\begin{lstlisting}[language=bash,frame=single]
# Количество запусков
awk 'END {print NR}' /var/log/text_analyzer/*.log

# Поиск ошибок
awk '/ERROR/' /var/log/text_analyzer/*.log
\end{lstlisting}

\section{Реакция на ошибки}

Если указанный файл не найден, программа выводит:
\begin{verbatim}
Ошибка: не удалось открыть файл 'notfound.txt'
\end{verbatim}
Код завершения: \texttt{-1}. Логируется ошибка с кодом \texttt{ERROR\_FILE\_NOT\_FOUND}.

\section{Состав модулей}

\begin{itemize}
    \item \texttt{core/analyzer.c} — ядро функций анализатора
    \item \texttt{ui/interface.c} — пользовательский интерфейс
    \item \texttt{utils/logger.c} — система логирования
    \item \texttt{main.c} — главный модуль с парсером аргументов
\end{itemize}

\section{Контакты}

\textbf{Автор:} Березин Евгений Андреевич \\
\textbf{Email:} trgchllw@mail.ru \\
\textbf{Группа:} ИТПМ-125 \\
\textbf{Университет:} РГУ им. Косыгина

