\documentclass[a4paper,12pt]{article}
\usepackage[utf8]{inputenc}
\usepackage[russian]{babel}
\usepackage{listings}
\usepackage{geometry}
\geometry{left=2.5cm,right=2.5cm,top=2cm,bottom=2cm}

\title{\textbf{Инструкция по анализу логов Text Analyzer}}
\author{Студент группы ИТПМ-125 \\ Березин Евгений Андреевич}
\date{\today}

\begin{document}

\maketitle

\section{Общие сведения}

Лог-файлы программы Text Analyzer хранятся в директории \texttt{/var/log/text\_analyzer/} и содержат информацию о всех операциях пользователей.

\section{Структура лога}

Каждая запись в логе имеет формат:
\begin{lstlisting}[frame=single]
[YYYY-MM-DD HH:MM:SS] user_id: action = result | Duration=Nс | Code=X
\end{lstlisting}

Где:
\begin{itemize}
    \item \texttt{user\_id} — идентификатор пользователя
    \item \texttt{action} — выполненное действие (например, count\_stats)
    \item \texttt{Duration} — продолжительность операции в секундах
    \item \texttt{Code} — код завершения (0 — успех, -1 — ошибка)
\end{itemize}

\section{Инструменты анализа}

Для анализа используется утилита \texttt{awk}.

\subsection{Подсчет общего количества запусков}
\begin{lstlisting}[language=bash,frame=single]
awk 'END {print "Всего запусков:", NR}' /var/log/text_analyzer/*.log
\end{lstlisting}

\subsection{Подсчет запусков по дням}
\begin{lstlisting}[language=bash,frame=single]
awk -F'[_ ]' '{print $1}' /var/log/text_analyzer/*.log | sort | uniq -c
\end{lstlisting}

\subsection{Подсчет запусков по часам}
\begin{lstlisting}[language=bash,frame=single]
awk -F'[ :]' '{print $2}' /var/log/text_analyzer/*.log | sort | uniq -c
\end{lstlisting}

\subsection{Поиск всех ошибок}
\begin{lstlisting}[language=bash,frame=single]
awk '/Code=-1/' /var/log/text_analyzer/*.log
\end{lstlisting}

\subsection{Статистика по обработанным файлам}
\begin{lstlisting}[language=bash,frame=single]
awk '/count_stats/ {stats++} END {print "Обработано файлов:", stats}' /var/log/text_analyzer/*.log
\end{lstlisting}

\subsection{Поиск по временному диапазону}
Анализ логов за конкретный день:
\begin{lstlisting}[language=bash,frame=single]
awk '/\[2025-12-19/' /var/log/text_analyzer/*.log
\end{lstlisting}

\section{Примеры использования}

\subsection{Ежедневный отчет}
Скрипт для генерации отчета за сутки:
\begin{lstlisting}[language=bash,frame=single]
#!/bin/bash
LOG_DIR="/var/log/text_analyzer"
DATE=$(date +%Y-%m-%d)

echo "=== Отчет за $DATE ==="
echo "Запусков:"
awk -v date="$DATE" '$0 ~ date' $LOG_DIR/*.log | wc -l

echo "Ошибок:"
awk -v date="$DATE" '$0 ~ date && /Code=-1/' $LOG_DIR/*.log | wc -l
\end{lstlisting}

\subsection{Мониторинг производительности}
Подсчет средней продолжительности операций:
\begin{lstlisting}[language=bash,frame=single]
awk -F'Duration=|s ' '{sum+=$2; count++} END {print "Среднее время:", sum/count, "сек"}' /var/log/text_analyzer/*.log
\end{lstlisting}

\section{Форматы лог-файлов}

\subsection{Именование файлов}
Имена файлов формируются по шаблону:
\begin{lstlisting}
YYYY-MM-DD_HH-MM-SS.log
\end{lstlisting}

Где:
\begin{itemize}
    \item \texttt{YYYY} — год
    \item \texttt{MM} — месяц
    \item \texttt{DD} — день
    \item \texttt{HH-MM-SS} — время создания
\end{itemize}

\subsection{Права доступа}
Права на файлы логов: \texttt{0644} (чтение/запись владельцу, чтение остальным).

\section{Хранение и ротация}

Логи не удаляются автоматически. Для управления используйте системные утилиты:

\begin{lstlisting}[language=bash,frame=single]
# Сжать старые логи
find /var/log/text_analyzer/ -name "*.log" -mtime +30 -exec gzip {} \;

# Удалить логи старше 90 дней
find /var/log/text_analyzer/ -name "*.log" -mtime +90 -delete
\end{lstlisting}

\section{Решение проблем}

\subsection{Логи не создаются}
\begin{enumerate}
    \item Проверьте права на директорию: \texttt{ls -ld /var/log/text\_analyzer}
    \item Убедитесь, что программа не запущена с ключом \texttt{-S} (тихий режим)
    \item Проверьте доступ к диску: \texttt{df -h /var/log}
\end{enumerate}

\subsection{Поврежденные логи}
Если лог поврежден, удалите его вручную:
\begin{lstlisting}[language=bash,frame=single]
sudo rm /var/log/text_analyzer/поврежденный_файл.log
\end{lstlisting}

\end{document}